\section{Conclusions and Future Work} \label{sec:conclusion}
In this paper, we presented a comprehensive system that optimizes the workflow of an incident commander when dealing with urban search \& rescue missions in cases where the initial reconnaissance of a collapsed structure is performed using unmanned robots. We analyzed the workflow from a visualization perspective and designed a linked multiple-view system that allows the \IC\ to effectively annotate a three-dimensional rendering of a filtered point cloud. Based on this enriched data, our system computes and analyzes a large number of paths. The resulting information is visualized for the \IC , who can then select a path that is optimal according to his experience and knowledge. To investigate the effectivity of the proposed system we have conducted a study with expert users. The resulting positive feedback makes us confident that the proposed system has the potential to improve future search \& rescue planning missions.

For future work we like to perform a thorough in-use evaluation workshop with domain experts using a real-world scenario. This will provide us with valuable direct feedback to improve the usability of the system. Furthermore, we will investigate the applicability of adaptive or stochastic sampling methods, like Monte-Carlo methods, on the high-dimensional parameter space for path computation.
