\section{Incident Commander Workflow} \label{sec:workflow}

\graphicspath{{figures/}}
\begin{figure}
\centering
\def\svgwidth{\columnwidth}
\section{Incident Commander Workflow} \label{sec:workflow}

\graphicspath{{figures/}}
\begin{figure}
\centering
\def\svgwidth{\columnwidth}
\section{Incident Commander Workflow} \label{sec:workflow}

\graphicspath{{figures/}}
\begin{figure}
\centering
\def\svgwidth{\columnwidth}
\section{Incident Commander Workflow} \label{sec:workflow}

\graphicspath{{figures/}}
\begin{figure}
\centering
\def\svgwidth{\columnwidth}
\input{figures/workflow.pdf_tex}
\caption{A schematic timeline overview of the currently employed workflow and our proposed system, showing all events (red) and actions (blue). Utilizing a parallel scanning (yellow) and thus faster and less dangerous exploration, we decrease the overall time-to-rescue.}
\label{fig:workflow:workflow}
\end{figure}

While there is no definition of a rescue team in an USAR incident, in most protocols, for example FEMA~\cite{fema08} or Emergency Management Australia~\cite{em35}, one \IC\ is responsible for a single building and instructs multiple rescue responders inside this building. We will describe the current workflow of the \IC\ first and then propose a visualization-enhanced workflow supported by our system (see Figure~\ref{fig:workflow:workflow}).

\noindent {\bfseries Current workflow.} After arriving at the disaster scene, the first step for the responders is to explore and secure the area outside the collapsed building. No rescuer is allowed to enter the building before it is secured, which can take up to an hour to finish. Then, based on the gathered information, the \IC\ determines valid entry points into the structure and directs rescuers into the collapsed structure to perform reconnaissance. Using constant radio communication, the rescuer inside the building slowly moves forward and reports his progress to the \IC\ who draws a two dimensional map based on that information, as shown in Figure~\ref{fig:workflow:sota}. The rescuer enters an unknown building with unquantifiable risks, like gas leaks or dormant fires, inside. Not only is the two dimensional drawing of an unstructured three dimensional building insufficient and inaccurate, it also does not provide acceptable spatial awareness. This is clearly an example of opportunistic control, where decisions are made opportunistically based on feedback from the environment. The map is made as the rescuer proceeds, inhibiting higher levels of control in the beginning of the path. Although responders may recognize situations, decisions regarding the path to take are limited to the extent of the current exploration and to their view of the local environment. At least initially, therefore, any RPD is restricted to the local environment. Global planning is limited further by the ability of responders to communicate relevant structural information accurately to the \IC.

\noindent {\bfseries Visualization-enhanced workflow.} The initial steps of arriving and exploring the area outside the collapsed structure are the same as in the currently employed workflow. While the responders secure the building, the most time-consuming of the initial tasks, the unmanned robots can be released into the structure and start recording and measuring the inside of the structure and feeding back the information to the \IC\ (the overlapping box in Figure~\ref{fig:workflow:workflow}). The map data from possibly multiple robots is co-registered and the \IC\ inspects the map and determines viable entry points into the structure. The robots' sensors are able to detect most signs of victims using thermal cameras and heart beat detectors~\cite{6027084, Wu12Eulerian}, but as these measurements are uncertain both false positives and false negatives might arise from the data. The same holds true for hazardous environments like fires, gas leaks, structural unsafe areas, chemical spills, or radiation. All data retrieval and preprocessing can be done in parallel while securing the perimeter, so that all information is available when the insertion begins. Based on suggested or selected POIs, the system computes optimal paths through the dataset which the \IC\ can use to direct the rescuers into the building, this time making informed decision about which ways to take, thus drastically reducing time-to-rescue as the rescuers do not have to explore the building to the same extent, but can proceed directly to the POIs. The proposed system applies RPD to the whole situation, extend the number of available cues, and extend planning from local conditions (tracking and regulating) to higher ECOM levels.

When planning the access paths, a variety of factors must be taken into account. The responder has to maintain a safe distance from hazardous environments, avoid overhanging structures, and the ground must be both stable and level. The uncertainty in the data, the varying requirements, and the required expert knowledge of the \IC\ make it unfeasible to derive an algorithm taking all these variables into account. Furthermore, as these variables are extracted from uncertain data, they are difficult to quantify. The \IC\ has to perform trade-offs to choose between alternatives, for example choosing a longer path, which is faster to traverse than a more dangerous, shorter path. These requirements for expert knowledge and decision making, as well as the uncertainty, make the path selection a perfect example demonstrating how the human in the loop can benefit the decision making.

While the \IC\ is instructing the rescuer to follow one chosen path, the rescuer feeds back new information, for example possible victims or new hazards, that he detects. The \IC\ incorporates this information into the application to update the mapping information and paths in real time. This is of high importance as features might not only have been missed by the robots, but detected features might change during the rescue operation. Fires can start or extinguish, subsequent structural collapses can make areas inaccessible, or debris is removed after the initial reconnaissance making previously inaccessible areas available. Although we are designing a system lifting the decision-making to a strategic control, this bears an opportunistic element of planning in the COCOM-sense. However, the system has to support replanning in a strategic control mode to be able to adapt to changing environments.

\caption{A schematic timeline overview of the currently employed workflow and our proposed system, showing all events (red) and actions (blue). Utilizing a parallel scanning (yellow) and thus faster and less dangerous exploration, we decrease the overall time-to-rescue.}
\label{fig:workflow:workflow}
\end{figure}

While there is no definition of a rescue team in an USAR incident, in most protocols, for example FEMA~\cite{fema08} or Emergency Management Australia~\cite{em35}, one \IC\ is responsible for a single building and instructs multiple rescue responders inside this building. We will describe the current workflow of the \IC\ first and then propose a visualization-enhanced workflow supported by our system (see Figure~\ref{fig:workflow:workflow}).

\noindent {\bfseries Current workflow.} After arriving at the disaster scene, the first step for the responders is to explore and secure the area outside the collapsed building. No rescuer is allowed to enter the building before it is secured, which can take up to an hour to finish. Then, based on the gathered information, the \IC\ determines valid entry points into the structure and directs rescuers into the collapsed structure to perform reconnaissance. Using constant radio communication, the rescuer inside the building slowly moves forward and reports his progress to the \IC\ who draws a two dimensional map based on that information, as shown in Figure~\ref{fig:workflow:sota}. The rescuer enters an unknown building with unquantifiable risks, like gas leaks or dormant fires, inside. Not only is the two dimensional drawing of an unstructured three dimensional building insufficient and inaccurate, it also does not provide acceptable spatial awareness. This is clearly an example of opportunistic control, where decisions are made opportunistically based on feedback from the environment. The map is made as the rescuer proceeds, inhibiting higher levels of control in the beginning of the path. Although responders may recognize situations, decisions regarding the path to take are limited to the extent of the current exploration and to their view of the local environment. At least initially, therefore, any RPD is restricted to the local environment. Global planning is limited further by the ability of responders to communicate relevant structural information accurately to the \IC.

\noindent {\bfseries Visualization-enhanced workflow.} The initial steps of arriving and exploring the area outside the collapsed structure are the same as in the currently employed workflow. While the responders secure the building, the most time-consuming of the initial tasks, the unmanned robots can be released into the structure and start recording and measuring the inside of the structure and feeding back the information to the \IC\ (the overlapping box in Figure~\ref{fig:workflow:workflow}). The map data from possibly multiple robots is co-registered and the \IC\ inspects the map and determines viable entry points into the structure. The robots' sensors are able to detect most signs of victims using thermal cameras and heart beat detectors~\cite{6027084, Wu12Eulerian}, but as these measurements are uncertain both false positives and false negatives might arise from the data. The same holds true for hazardous environments like fires, gas leaks, structural unsafe areas, chemical spills, or radiation. All data retrieval and preprocessing can be done in parallel while securing the perimeter, so that all information is available when the insertion begins. Based on suggested or selected POIs, the system computes optimal paths through the dataset which the \IC\ can use to direct the rescuers into the building, this time making informed decision about which ways to take, thus drastically reducing time-to-rescue as the rescuers do not have to explore the building to the same extent, but can proceed directly to the POIs. The proposed system applies RPD to the whole situation, extend the number of available cues, and extend planning from local conditions (tracking and regulating) to higher ECOM levels.

When planning the access paths, a variety of factors must be taken into account. The responder has to maintain a safe distance from hazardous environments, avoid overhanging structures, and the ground must be both stable and level. The uncertainty in the data, the varying requirements, and the required expert knowledge of the \IC\ make it unfeasible to derive an algorithm taking all these variables into account. Furthermore, as these variables are extracted from uncertain data, they are difficult to quantify. The \IC\ has to perform trade-offs to choose between alternatives, for example choosing a longer path, which is faster to traverse than a more dangerous, shorter path. These requirements for expert knowledge and decision making, as well as the uncertainty, make the path selection a perfect example demonstrating how the human in the loop can benefit the decision making.

While the \IC\ is instructing the rescuer to follow one chosen path, the rescuer feeds back new information, for example possible victims or new hazards, that he detects. The \IC\ incorporates this information into the application to update the mapping information and paths in real time. This is of high importance as features might not only have been missed by the robots, but detected features might change during the rescue operation. Fires can start or extinguish, subsequent structural collapses can make areas inaccessible, or debris is removed after the initial reconnaissance making previously inaccessible areas available. Although we are designing a system lifting the decision-making to a strategic control, this bears an opportunistic element of planning in the COCOM-sense. However, the system has to support replanning in a strategic control mode to be able to adapt to changing environments.

\caption{A schematic timeline overview of the currently employed workflow and our proposed system, showing all events (red) and actions (blue). Utilizing a parallel scanning (yellow) and thus faster and less dangerous exploration, we decrease the overall time-to-rescue.}
\label{fig:workflow:workflow}
\end{figure}

While there is no definition of a rescue team in an USAR incident, in most protocols, for example FEMA~\cite{fema08} or Emergency Management Australia~\cite{em35}, one \IC\ is responsible for a single building and instructs multiple rescue responders inside this building. We will describe the current workflow of the \IC\ first and then propose a visualization-enhanced workflow supported by our system (see Figure~\ref{fig:workflow:workflow}).

\noindent {\bfseries Current workflow.} After arriving at the disaster scene, the first step for the responders is to explore and secure the area outside the collapsed building. No rescuer is allowed to enter the building before it is secured, which can take up to an hour to finish. Then, based on the gathered information, the \IC\ determines valid entry points into the structure and directs rescuers into the collapsed structure to perform reconnaissance. Using constant radio communication, the rescuer inside the building slowly moves forward and reports his progress to the \IC\ who draws a two dimensional map based on that information, as shown in Figure~\ref{fig:workflow:sota}. The rescuer enters an unknown building with unquantifiable risks, like gas leaks or dormant fires, inside. Not only is the two dimensional drawing of an unstructured three dimensional building insufficient and inaccurate, it also does not provide acceptable spatial awareness. This is clearly an example of opportunistic control, where decisions are made opportunistically based on feedback from the environment. The map is made as the rescuer proceeds, inhibiting higher levels of control in the beginning of the path. Although responders may recognize situations, decisions regarding the path to take are limited to the extent of the current exploration and to their view of the local environment. At least initially, therefore, any RPD is restricted to the local environment. Global planning is limited further by the ability of responders to communicate relevant structural information accurately to the \IC.

\noindent {\bfseries Visualization-enhanced workflow.} The initial steps of arriving and exploring the area outside the collapsed structure are the same as in the currently employed workflow. While the responders secure the building, the most time-consuming of the initial tasks, the unmanned robots can be released into the structure and start recording and measuring the inside of the structure and feeding back the information to the \IC\ (the overlapping box in Figure~\ref{fig:workflow:workflow}). The map data from possibly multiple robots is co-registered and the \IC\ inspects the map and determines viable entry points into the structure. The robots' sensors are able to detect most signs of victims using thermal cameras and heart beat detectors~\cite{6027084, Wu12Eulerian}, but as these measurements are uncertain both false positives and false negatives might arise from the data. The same holds true for hazardous environments like fires, gas leaks, structural unsafe areas, chemical spills, or radiation. All data retrieval and preprocessing can be done in parallel while securing the perimeter, so that all information is available when the insertion begins. Based on suggested or selected POIs, the system computes optimal paths through the dataset which the \IC\ can use to direct the rescuers into the building, this time making informed decision about which ways to take, thus drastically reducing time-to-rescue as the rescuers do not have to explore the building to the same extent, but can proceed directly to the POIs. The proposed system applies RPD to the whole situation, extend the number of available cues, and extend planning from local conditions (tracking and regulating) to higher ECOM levels.

When planning the access paths, a variety of factors must be taken into account. The responder has to maintain a safe distance from hazardous environments, avoid overhanging structures, and the ground must be both stable and level. The uncertainty in the data, the varying requirements, and the required expert knowledge of the \IC\ make it unfeasible to derive an algorithm taking all these variables into account. Furthermore, as these variables are extracted from uncertain data, they are difficult to quantify. The \IC\ has to perform trade-offs to choose between alternatives, for example choosing a longer path, which is faster to traverse than a more dangerous, shorter path. These requirements for expert knowledge and decision making, as well as the uncertainty, make the path selection a perfect example demonstrating how the human in the loop can benefit the decision making.

While the \IC\ is instructing the rescuer to follow one chosen path, the rescuer feeds back new information, for example possible victims or new hazards, that he detects. The \IC\ incorporates this information into the application to update the mapping information and paths in real time. This is of high importance as features might not only have been missed by the robots, but detected features might change during the rescue operation. Fires can start or extinguish, subsequent structural collapses can make areas inaccessible, or debris is removed after the initial reconnaissance making previously inaccessible areas available. Although we are designing a system lifting the decision-making to a strategic control, this bears an opportunistic element of planning in the COCOM-sense. However, the system has to support replanning in a strategic control mode to be able to adapt to changing environments.

\caption{A schematic timeline overview of the currently employed workflow and our proposed system, showing all events (red) and actions (blue). Utilizing a parallel scanning (yellow) and thus faster and less dangerous exploration, we decrease the overall time-to-rescue.}
\label{fig:workflow:workflow}
\end{figure}

While there is no definition of a rescue team in an USAR incident, in most protocols, for example FEMA~\cite{fema08} or Emergency Management Australia~\cite{em35}, one \IC\ is responsible for a single building and instructs multiple rescue responders inside this building. We will describe the current workflow of the \IC\ first and then propose a visualization-enhanced workflow supported by our system (see Figure~\ref{fig:workflow:workflow}).

\noindent {\bfseries Current workflow.} After arriving at the disaster scene, the first step for the responders is to explore and secure the area outside the collapsed building. No rescuer is allowed to enter the building before it is secured, which can take up to an hour to finish. Then, based on the gathered information, the \IC\ determines valid entry points into the structure and directs rescuers into the collapsed structure to perform reconnaissance. Using constant radio communication, the rescuer inside the building slowly moves forward and reports his progress to the \IC\ who draws a two dimensional map based on that information, as shown in Figure~\ref{fig:workflow:sota}. The rescuer enters an unknown building with unquantifiable risks, like gas leaks or dormant fires, inside. Not only is the two dimensional drawing of an unstructured three dimensional building insufficient and inaccurate, it also does not provide acceptable spatial awareness. This is clearly an example of opportunistic control, where decisions are made opportunistically based on feedback from the environment. The map is made as the rescuer proceeds, inhibiting higher levels of control in the beginning of the path. Although responders may recognize situations, decisions regarding the path to take are limited to the extent of the current exploration and to their view of the local environment. At least initially, therefore, any RPD is restricted to the local environment. Global planning is limited further by the ability of responders to communicate relevant structural information accurately to the \IC.

\noindent {\bfseries Visualization-enhanced workflow.} The initial steps of arriving and exploring the area outside the collapsed structure are the same as in the currently employed workflow. While the responders secure the building, the most time-consuming of the initial tasks, the unmanned robots can be released into the structure and start recording and measuring the inside of the structure and feeding back the information to the \IC\ (the overlapping box in Figure~\ref{fig:workflow:workflow}). The map data from possibly multiple robots is co-registered and the \IC\ inspects the map and determines viable entry points into the structure. The robots' sensors are able to detect most signs of victims using thermal cameras and heart beat detectors~\cite{6027084, Wu12Eulerian}, but as these measurements are uncertain both false positives and false negatives might arise from the data. The same holds true for hazardous environments like fires, gas leaks, structural unsafe areas, chemical spills, or radiation. All data retrieval and preprocessing can be done in parallel while securing the perimeter, so that all information is available when the insertion begins. Based on suggested or selected POIs, the system computes optimal paths through the dataset which the \IC\ can use to direct the rescuers into the building, this time making informed decision about which ways to take, thus drastically reducing time-to-rescue as the rescuers do not have to explore the building to the same extent, but can proceed directly to the POIs. The proposed system applies RPD to the whole situation, extend the number of available cues, and extend planning from local conditions (tracking and regulating) to higher ECOM levels.

When planning the access paths, a variety of factors must be taken into account. The responder has to maintain a safe distance from hazardous environments, avoid overhanging structures, and the ground must be both stable and level. The uncertainty in the data, the varying requirements, and the required expert knowledge of the \IC\ make it unfeasible to derive an algorithm taking all these variables into account. Furthermore, as these variables are extracted from uncertain data, they are difficult to quantify. The \IC\ has to perform trade-offs to choose between alternatives, for example choosing a longer path, which is faster to traverse than a more dangerous, shorter path. These requirements for expert knowledge and decision making, as well as the uncertainty, make the path selection a perfect example demonstrating how the human in the loop can benefit the decision making.

While the \IC\ is instructing the rescuer to follow one chosen path, the rescuer feeds back new information, for example possible victims or new hazards, that he detects. The \IC\ incorporates this information into the application to update the mapping information and paths in real time. This is of high importance as features might not only have been missed by the robots, but detected features might change during the rescue operation. Fires can start or extinguish, subsequent structural collapses can make areas inaccessible, or debris is removed after the initial reconnaissance making previously inaccessible areas available. Although we are designing a system lifting the decision-making to a strategic control, this bears an opportunistic element of planning in the COCOM-sense. However, the system has to support replanning in a strategic control mode to be able to adapt to changing environments.
